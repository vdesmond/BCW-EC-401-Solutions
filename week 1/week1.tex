\documentclass[a4paper]{article}
\usepackage[document]{ragged2e}
%\usepackage{simplemargins}

%\usepackage[square]{natbib}
\usepackage{amsmath}
\usepackage{amsfonts}
\usepackage{amssymb}
\usepackage{graphicx}
\usepackage{tabularx}
\usepackage[numbered,framed]{matlab-prettifier}
\usepackage{minted}
\usepackage{svg}
\usepackage{graphicx}
\usepackage{fancyhdr}
\pagestyle{fancy}
\usepackage{xcolor}
\usepackage{lipsum}
\setlength\headheight{26pt} %% just to make warning go away. Adjust the value after looking into the warning.
\lfoot{{\small\textsf{\thepage}}}
\rfoot{\small\textsf{\today}}
\cfoot{}
\rhead{\includegraphics[width=50pt]{bigcat}}
\lhead{\includegraphics[width=30pt]{ssn}}
\begin{document}
% \pagenumbering{gobble}

\Large
 \begin{center}
BigCat Wireless - EC401\\
\textbf{Assignment 1}

\hspace{10pt}

% Author names and affiliations
\large
Vignesh Mohan\\
\smallskip
\small
183002181\\
\small
Electronics and Communication Department\\
\small
Sri Sivasubramiya Nadar College of Engineering, 603110\\
\end{center}

\hspace{10pt}
\normalsize

\textit{1. List the differences between wired and wireless communication?}\\
\bigskip
\begin{tabular}{| p{0.5\linewidth} | p{0.5\linewidth} |}
 \hline
 \textbf{Wired Communications} & \textbf{Wireless Communications} \\ 
 \hline
 In a wired communication system there exists a physical medium (like wires) & In wireless communication systems, there does not exist any physical medium\\  
 \hline
 In such systems, information is sent in form of electrical signals (or optical signals) & Information is transmitted using electromagnetic waves.
 \\
  \hline
   Collision Detection is used & Collision Avoidance is used \\
\hline
\end{tabular}

\bigskip
\textit{2. Perform addition of -1.895[S(1,7)] + 0.125[S(0,3)]}\\
\bigskip
Operand 1 (O\textsubscript1) = S(1,7) = -1.895\\
Operand 2 (O\textsubscript2) = S(0,3) = 0.125\\
Resultant bit growth = S(max(1,0)+1,max(7,3)) = S(2,7)\\
\medskip
Scaling, we get:
\begin{gather*} 
-1.895 * 2^7 = -242 = 100001110 \;\; (2's complement)\\
0.125	* 2^7 = 16 = 000010000
\end{gather*}
\vspace{-20pt}
\begin{align*} 
-1.895[S(1,7)] + 0.125[S(0,3)] =\\ 
100001110\\
(+)  \;\;000010000\\
\rule{100pt}{0.1pt}\\
0\hspace{10pt}100011110\\
====> -226
\end{align*}
Scaling back, we get:
\begin{gather*} 
-226/2^7 = -1.765625
\end{gather*}
Therefore, -1.895[S(1,7)] + 0.125[S(0,3)] = -1.765625

\bigskip
\textit{3. Perform multiplication of -0.125[S(0,3)] * -0.5[S(0,3)] }\\
\bigskip
Operand 1 (O\textsubscript1) = S(0,3) = -0.125\\
Operand 2 (O\textsubscript2) = S(0,3) = -0.5\\
Resultant bit growth = S(0+0+1,3+3) = S(1,6)\\
\medskip
Scaling, we get:
\begin{gather*} 
-0.125 * 2^3 = -1 = 1111 \;\; (2's complement)\\
-0.5 * 2^3 = -4 = 1100 \;\; (2's complement)\\
\end{gather*}
\vspace{-20pt}
\begin{align*} 
-0.125[S(0,3)] * -0.5[S(0,3)] =\\ 
1111\\
(*)  \;\;1100\\
\rule{100pt}{0.1pt}\\
0000\\
0000\;\\
1111\;\;\\
1111\;\;\;\\
\rule{100pt}{0.1pt}\\
01011 0100\\
====> 4
\end{align*}
Scaling back, we get:
\begin{gather*} 
4/2^6 = 0.0625
\end{gather*}
Therefore, -0.125[S(0,3)] * -0.5[S(0,3)] = 0.0625

\bigskip
\textit{4. Represent the number 11101100 S(1,6) in result format S(0,3)}\\
\bigskip

Rounding off 11101100 S(1,6) to result format S(0,3) =
\begin{align*} 
11101100\\
(+)  \;\;00001000\\
\rule{100pt}{0.1pt}\\
11110100\\
\end{align*}
After Truncation, the result in S(0,3) would be 1111 = -0.125\\
\bigskip
\textit{5. Generate 16-QAM modulated signal without using inbuilt Matlab functions and plot the same using rectangular plane (IQ-Plot)}\\
\bigskip
A seperate function for 16 QAM modulation is defined and used to modulate a random signal. The constellation diagram obtained is also shown.
\definecolor{bg}{rgb}{0.95,0.95,0.95}
\begin{minted}[linenos=true,bgcolor=bg]{matlab}
% main.m
M = 16;
x = randi([0 M - 1], 100, 1);
y = qammod16(x);
scatterplot(y);
\end{minted}   
\newpage
\begin{minted}[linenos=true,bgcolor=bg]{matlab}
% qammod.m
function y = qammod16(x)
len = length(x);
y = zeros(len, 1);
for i = 1:len
    switch x(i)
        case 0
            y(i) = - 3.0 + 1.0i;
        case 1
            y(i) = - 3.0 - 3.0i;
        case 2
            y(i) = - 3.0 - 1.0i;
        case 3
            y(i) = - 1.0 + 3.0i;
        case 4
            y(i) = - 3.0 + 3.0i;
        case 5
            y(i) = - 1.0 + 1.0i;
        case 6
            y(i) = - 1.0 - 3.0i;
        case 7
            y(i) = - 1.0 - 1.0i;
        case 8
            y(i) = 3.0 + 3.0i;
        case 9
            y(i) = 3.0 + 1.0i;
        case 10
            y(i) = 3.0 - 3.0i;
        case 11
            y(i) = 3.0 - 1.0i;
        case 12
            y(i) = 1.0 + 3.0i;
        case 13
            y(i) = 1.0 + 1.0i;
        case 14
            y(i) = 1.0 - 3.0i;
        case 15
            y(i) = 1.0 - 1.0i;
    end
end
end
\end{minted}
\newpage
\begin{figure}[htbp]
  \centering
  \includesvg{const16}
  \caption{Constellation Diagram - 16 QAM}
\end{figure}
\end{document}