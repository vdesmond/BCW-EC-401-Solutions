\documentclass[a4paper]{article}
\usepackage[document]{ragged2e}
%\usepackage{simplemargins}

%\usepackage[square]{natbib}
\usepackage{mathtools}
\usepackage{amsfonts}
\usepackage{amssymb}
\usepackage{graphicx}
\usepackage{tabularx}
\usepackage[export]{adjustbox}
\usepackage[numbered,framed]{matlab-prettifier}
\usepackage{minted}
\usepackage{svg}
\usepackage{graphicx}
\usepackage{fancyhdr}
\pagestyle{fancy}
\usepackage{xcolor}
\usepackage{lipsum}
\setlength\headheight{26pt} %% just to make warning go away. Adjust the value after looking into the warning.
\lfoot{{\small\textsf{\thepage}}}
\rfoot{\small\textsf{\today}}
\cfoot{}
\rhead{\includegraphics[width=50pt]{bigcat}}
\lhead{\includegraphics[width=30pt]{ssn}}
\begin{document}
	% \pagenumbering{gobble}
	
	\Large
	\begin{center}
		BigCat Wireless - EC401\\
		\textbf{Assignment 4}
		
		\hspace{10pt}
		
		% Author names and affiliations
		\large
		Vignesh Mohan\\
		\smallskip
		\small
		183002181\\
		\small
		Electronics and Communication Department\\
		\small
		Sri Sivasubramiya Nadar College of Engineering, 603110\\
	\end{center}
	
	\hspace{10pt}
	\normalsize
	
	\textit{1. Calculate bandwidth required to live stream a cricket match with Full HD resolution (1920x1080 pixels) in raw format at 30 fps (frames/sec) using OFDM modulation and 16-QAM symbol mapping. Assume we are going to use 1024-point IFFT (Hint: in a colour video each pixel is represented with 3 Bytes)}
	\begin{align*}
	\intertext{1920 x 1080 pixels with each pixel represented by 3 bytes = 24 bits: }
	1\;s & = 30 \;frames\\
	&   = 30 * 1920 * 1080 * 24\; bits\\
	&   = 1492992000\; bits\\
	&  = 373248000 \;symbols \;(16-QAM \;symbol = 4 \; bits)\\
	\intertext{Since we use 1024-point IFFT, number of subcarriers = 1024}
	1\;s & = \frac{373248000}{1024}\\
	& = 364500 \;OFDM \;symbols\\
	\shortintertext{For 1 symbol, time period is 2.74 $\mu s$}
	\intertext{Since we use 1024-point IFFT, number of subcarriers = 1024}
	BW & = 364.5 \;KHz * 1024\\
	Required\;Bandwidth & = 373.24 \; MHz\\
	\end{align*}
	
	\bigskip
	\textit{2. Calculate the duration of cyclic prefix required to overcome multipath time spreading in a OFDM communication system with input bit rate of 5 Mbps and uses 512 pt IFFT. Based on analysis of wireless channel its has been found that the worst multipath component is 6.14 km.}
	\begin{align*}
	\shortintertext{Input bit rate = 5Mbps}
	\shortintertext{Number of subcarriers = 512}
	\shortintertext{Worst case distance = 6.14 km}
	\shortintertext{Multipath time spreading = $\frac{6.14*10^3}{3*10^8} = 20.4 \mu s$}
	\shortintertext{Thus guard interval must be greater than $20.4 \mu s$}
	\intertext{Assumimg 16 QAM symbols are used, we get}
	1 \;s & = 5 * 10^6 bits\\
	& = 1.25 * 10^6 QAM symbols\\
	& = \frac{1.25 * 10^6}{512} OFDM symbols\\
	& = 2.44 * 10^3 OFDM symbols\\
	\shortintertext{For 1 symbol, time period is 0.41 $\mu s$}
	\shortintertext{For cyclic prefix, we take  $\frac{1}{10}$ of symbol time, which is 0.041 ms}
	\shortintertext{Guard interval (0.041 ms) is greater than multipath time spread (0.0204 ms)}
	\end{align*}
	
	\textit{3. Create an OFDM system: Use MATLAB or Python code only to create (no Simulink) Calculate the following:\\
		• Bit Error Rate\\
		Plot the following figures:\\
		• Transmitted spectrum\\
		• Received Constellation\\
	}
	\smallskip
	\definecolor{bg}{rgb}{0.95,0.95,0.95}
	\begin{minted}[linenos=true,bgcolor=bg]{matlab}
	% ofdm_simulation.m
	M = 16; %Modulation order
	k = log2(M); %Number of bits per symbol
	n = 51200; %Number of data samples
	
	data = randi([0 1], n, 1); %n samples
	
	dataInput = reshape(data, length(data)/k, k);
	dataSym = bi2de(dataInput);
	dataMod = qammod(dataSym,M);
	
	dataParallel = reshape(dataMod, 512, []); %Serial to parallel
	dataOFDMmod = ifft(dataParallel, 512, 1);  %IFFT
	txData = reshape(dataOFDMmod, [], 1);
	dataChannel = awgn(txData, 18, 'measured'); %SNR = 18 dB
	rxData = reshape(dataChannel, 512, []);
	dataOFDMdemod = fft(rxData, 512, 1); %FFT
	dataSerial = reshape(dataOFDMdemod, 1, []); %Parallel to Serial
	scatterplot(dataSerial);
	dataDemod = qamdemod(dataSerial, M)';
	
	%calculate BER
	recdataSym = de2bi(dataDemod);
	recdata = reshape(recdataSym,[], 1);
	[numerr, ber] = biterr(data, recdata);
	fprintf("Bit Error Rate = %.6f\n", ber)
	
	%Spectrum of Tx Signal
	fs = 20000000;
	[Pxx,W] = pwelch(txData',[],[],4096,fs);    
	figure;
	plot([-2048:2047]*fs/4096,10*log10(fftshift(Pxx)),'b'); 
	grid;hold on;
	xlabel('Frequency (MHz)')
	ylabel('Power Spectral Density')
	hold off;
	\end{minted}
	Output:\\
	Bit Error Rate = 0.000117
	\begin{figure}[hbt!]
		\centering
		\includesvg[width=\linewidth]{ofdm_qammod}
		\caption{Constellation Diagram of received signal}
	\end{figure}
	\begin{figure}[hbt!]
		\centering
		\includesvg[width=\linewidth]{ofdm_spectrum}
		\caption{Spectrum (PSD) of transmitted signal (Note spectrum shown is without any subcarrier modulation)}
	\end{figure}
\end{document}