\documentclass[a4paper]{article}
\usepackage[document]{ragged2e}
%\usepackage{simplemargins}

%\usepackage[square]{natbib}
\usepackage{amsmath}
\usepackage{amsfonts}
\usepackage{amssymb}
\usepackage{graphicx}
\usepackage{tabularx}
\usepackage[export]{adjustbox}
\usepackage[numbered,framed]{matlab-prettifier}
\usepackage{minted}
\usepackage{svg}
\usepackage{graphicx}
\usepackage{fancyhdr}
\pagestyle{fancy}
\usepackage{xcolor}
\usepackage{lipsum}
\setlength\headheight{26pt} %% just to make warning go away. Adjust the value after looking into the warning.
\lfoot{{\small\textsf{\thepage}}}
\rfoot{\small\textsf{\today}}
\cfoot{}
\rhead{\includegraphics[width=50pt]{bigcat}}
\lhead{\includegraphics[width=30pt]{ssn}}
\begin{document}
% \pagenumbering{gobble}

\Large
 \begin{center}
BigCat Wireless - EC401\\
\textbf{Assignment 3}

\hspace{10pt}

% Author names and affiliations
\large
Vignesh Mohan\\
\smallskip
\small
183002181\\
\small
Electronics and Communication Department\\
\small
Sri Sivasubramiya Nadar College of Engineering, 603110\\
\end{center}

\hspace{10pt}
\normalsize

\textit{1. From the constellation plot shown in figure below, answer the following questions?}\\
\bigskip
\begin{figure}[hbt!]
  \centering
  \includegraphics[width=0.5\linewidth]{8PSK}
  \caption{Constellation plot which is to be identified}
\end{figure}
\begin{enumerate}
  \item \textit{What type of modulation does this represent?}\\
  The modulation is 8-PSK\\
  \item \textit{How many symbols (M) are represented?}\\
  8 symbols
  \item \textit{How many bits per symbol (N) are used?}\\
  3 bits
  \item \textit{What’s the bit rate when baud rate is 10,000 symbols/second?}\\
  30,000 bits/second (30kbps)
  \item \textit{Compare the noise performance of this modulation with 16-QAM, which is better?}\\
  8-PSK should in general be less susceptible to noise because the symbols would be spread further apart. Approximately, the minimum distance for constellation points for 8-PSK is  $1.325\sqrt{E_b}$ which is slightly greater than for 16-QAM which is $1.265\sqrt{E_b}$.
  \item \textit{Compare the spectral efficiency of this modulation with QPSK, which is better?}\\
  8-PSK has better spectral efficiency since it uses $2R_b/3$ bandwidth which is less than the bandwidth $R_b$ used by QPSK
\end{enumerate}

\bigskip
\textit{2. Solve the IQ demodulator equation showing how I and Q components are received back.}
\begin{gather*}
\intertext{Composite Input signal: }
X(t) = I*cos(2\pi f_c t) - Q*sin(2\pi f_c t)
\intertext{Inphase Component (multipled with LO signal): }
X_i(t) = (I*cos(2\pi f_c t) - Q*sin(2\pi f_c t))*cos(2\pi f_c t)\\
    = \frac{I}{2} + \frac{I}{2}*cos(2\pi f_c t) - \frac{Q}{2}*sin(2\pi f_c t)
\intertext{Quadrature Component (multipled with $\pi/2$ shifted LO signal): }
X_q(t) = (I*cos(2\pi f_c t) - Q*sin(2\pi f_c t))* -sin(2\pi f_c t)\\
    = \frac{Q}{2} - \frac{Q}{2}*cos(2\pi f_c t) - \frac{I}{2}*sin(2\pi f_c t)
\intertext{After LPF (sin,cos terms with $f_c$ are cancelled ): }
X_i(t) = \frac{I}{2}\\
X_1(t) = \frac{Q}{2}\\
\end{gather*}

\bigskip
\textit{3. Calculate maximum theoretical spectral efficiency for:}\\
\begin{enumerate}
    \item \textit{$\pi/4$ QPSK:}\\
    Maximum theoretical spectral efficiency = 2 bits/s/Hz.
    \item \textit{256-QAM:}\\
    Maximum theoretical spectral efficiency = 8 bits/s/Hz.
\end{enumerate}
\bigskip

\newpage
\textit{4. Create a transmitter and receiver architecture.\\
Use the following parameters:\\
Input bit rate 1 Mbps\\
Modulation method: QPSK or 16-QAM\\
RRC roll-off factor 0.2\\
RRC output samples per symbol: 8\\
Carrier frequency 2.5MHz\\
Use AWGN channel with SNR =10dB\\
Equivalent settings in receiver side}\\
\bigskip
\definecolor{bg}{rgb}{0.95,0.95,0.95}
\begin{minted}[linenos=true,bgcolor=bg]{matlab}
% passband_modulation.m
close all;
bitrate = 1e06;
Fc = 2.5e06;
Fs = 8 * bitrate;

M = 4; % Modulation order
k = log2(M); % Bits/symbol
n = 500; % Transmitted bits
sps = 8; % Samples per symbol
EbNo = 10; % Eb/No (dB)

span = 8; % Filter span in symbols
rolloff = 0.20; % Rolloff factor

txfilter = comm.RaisedCosineTransmitFilter('RolloffFactor',
rolloff, 'FilterSpanInSymbols', span, ...
  'OutputSamplesPerSymbol', sps);

rxfilter = comm.RaisedCosineReceiveFilter('RolloffFactor',
rolloff, 'FilterSpanInSymbols', span, ...
  'InputSamplesPerSymbol', sps, ...
  'DecimationFactor', sps, 'Gain', 2);

ri = comm.internal.RandomIntegerGenerator('SetSize', M, ...
  'SampleTime', 1 / bitrate, 'SamplesPerFrame', n);

tx_lo = dsp.SineWave(1, Fc, 0, 'ComplexOutput', true, ...
  'SampleRate', Fs, 'SamplesPerFrame', sps * n);

rx_lo = dsp.SineWave(1, Fc, 0, 'ComplexOutput', true, ...
  'SampleRate', Fs, 'SamplesPerFrame', sps * n);

filtDelay = (txfilter.FilterSpanInSymbols + ...
  rxfilter.FilterSpanInSymbols) / 2;
errorRate = comm.ErrorRate('ReceiveDelay', filtDelay);

delay = dsp.Delay(8);
evm = comm.EVM();

biterr = 0;
totalbits = 0;
rmsEVM = 0;
tx_scope = dsp.SpectrumAnalyzer('SpectrumType', "Power Density", ...
  'SampleRate', Fs, 'FrequencyResolutionMethod', ...
  "WindowLength", 'WindowLength', 800, ...
  'PlotAsTwoSidedSpectrum', false);

rx_scope = dsp.SpectrumAnalyzer('SpectrumType', "Power", ...
  'SampleRate', Fs, 'FrequencyResolutionMethod', ...
  "WindowLength", 'WindowLength', 1024);

txSigall = [];
rxSigall = [];

tx_const_diag = scatterplot(ri());
rx_const_diag = scatterplot(ri());

for idx = 1:20
  dataIn = ri();
  modSig = pskmod(dataIn, 4, pi / 4);
  txFilterSig = txfilter(modSig);
  tx_carrier = tx_lo();
  txSig = real(txFilterSig .* tx_carrier);
  txSigall = txSig;

  SNR = EbNo + 10 * log10(k) - 10 * log10(sps);
  noisySig = awgn(txSig, SNR, 'measured');

  rx_wave = conj(rx_lo());
  rxSig = noisySig .* rx_wave;
  rxSigall = rxSig;

  rxFilterSig = rxfilter(rxSig);

  dataOut = pskdemod(rxFilterSig, 4, pi / 4);

  errStat = errorRate(dataIn, dataOut);
  biterr = biterr + errStat(2);
  totalbits = totalbits + errStat(3);

  rmsEVM = (rmsEVM + evm(delay(modSig), rxFilterSig)) / 2;

  scatterplot(modSig, 1, 0, 'y.', tx_const_diag);
  scatterplot(rxFilterSig, 1, 0, 'y.', rx_const_diag);
  tx_scope(txSigall);
  rx_scope(rxSigall);
end

fprintf('\nBit Errors = %d', biterr);
fprintf('\nBits Transmitted = %d\n', totalbits);
fprintf('\nBER = %5.2e\n', biterr / totalbits);
fprintf('\nRMS EVM = %.2f %% \n', rmsEVM);

release(tx_scope);
release(rx_scope);
\end{minted}

Sample output:\\

Bit Errors = 157
Bits Transmitted = 104840

BER = 1.50e-03

RMS EVM = 31.68 % 

\begin{figure}[hbt!]
    \centering
  \includesvg[width=0.8\linewidth]{tx_const}
  \caption{Transmitted constellation for QPSK}
\end{figure}
\begin{figure}[hbt!]
    \centering
  \includesvg[width=0.8\linewidth]{rx_const}
  \caption{Received constellation for QPSK}
\end{figure}
\begin{figure}[hbt!]
    \centering
  \includesvg[height=10cm, width=15cm]{transmitter}
  \caption{Transmitted spectrum for QPSK}
\end{figure}
\begin{figure}[hbt!]
    \centering
  \includesvg[height=10cm, width=15cm]{receiver}
  \caption{Received spectrum for QPSK}
\end{figure}
\end{document}